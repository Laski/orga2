\section{Introducción}

El siguiente informe se presenta como complemento y documentación del trabajo
practico final de la materia Organización del computador II.\\

El mismo fue realizado dentro del programa Mozilla Winter Of Security
2014\footnote{https://wiki.mozilla.org/Security/Automation/WinterOfSecurity2014}.
Éste tiene como finalidad introducir a alumnos de carreras afines a la
computación al mundo de la seguridad informática y la automatización. Para
esto, el equipo de Security
Automation\footnote{https://wiki.mozilla.org/Security/Automation} de Mozilla
propuso una serie de proyectos para que distintos grupos de alumnos pudieran
encarar, contando con un tutor provisto por ellos.\\

En este trabajo se llevó a cabo uno de dichos proyectos, que consistió en
proveer de funcionalidades de inspección de memoria a la plataforma de análisis
forense en tiempo real y respuesta a incidentes Mozilla
Investigator\footnote{http://mig.mozilla.org}, la cual será introducida a
continuación.\\

\subsection{Mozilla Investigator}

Mozilla Investigator surge en respuesta a los cambios que se ha vivido en la
infraestructura con el advenimiento del cloud computing. Mientras que
antes los distintos servidores y servicios permanecían estáticos a lo largo del
tiempo, sujetos a revisiones lentas y estrictas a la hora de ser alterados, hoy
en día la virtualización permite un dinamismo y escala en la infraestructura
que solía ser impensable. Este nuevo modelo demostró ser adecuado en un sin
numero de situaciones, pero no viene sin traer sus nuevos desafíos.\\

Uno de estos desafíos, y del cual Mozilla Investigator (MIG) se encarga, es
como manejar la seguridad y respuesta a incidentes en escalas nunca antes
vistas de manera rápida y efectiva.\\

Para llevar a cabo esta tarea, MIG provee un sistema distribuido que facilita
realizar distintos tipos de chequeos en los servidores de Mozilla. Esto se
logra con una arquitectura distribuida, en la cual los distintos servidores
cuentan con un agente corriendo de manera constante el cual lee una cola de
queries a medida que esta se va llenando. Una vez que un
query es leído, el agente actúa de manera acorde y envía su respuesta.
El funcionamiento del sistema es orquestado por MIG Scheduler, uno o
más servidores distinguidos, encargados de distribuir los queries a
los agentes y comunicar las respuestas al usuario. De este modo un investigador
de Mozilla puede conectarse al servidor de MIG y realizar consultas a miles de
servidores en simultaneo en cuestión de segundos, permitiendo acelerar la
reacción frente a incidentes de seguridad y agilizando las actividades
cotidianas de security operations.\\

Si bien MIG es usado hoy en día en producción, se encuentra aun bajo
desarrollo, principalmente con respecto a qué tipo de queries puede
responder un agente. En la actualidad esto está limitado a responder consultas
sobre la integridad del sistema de archivos, como chequear el nombre, contenido
y hash de los mismos; y consultar el estado del stack de red del sistema donde
está corriendo.\\

Existen proyectos para agregar soporte para distintos tipos de queries, como
consultar y manejar reglas de los distintos firewalls, crear, bloquear y
destruir cuentas de usuarios, analizar el trafico de la red, acceso a bajo
nivel de los distintos componentes físicos del servidor, y muchos otros. En
particular, uno de estos módulos, que pronto se encontrará en producción, es el
desarrollo para este trabajo, el cual permite analizar la memoria principal de
los procesos que se encuentran corriendo en el sistema, y que se introducirá a
continuación.\\

\subsection{MASCHE}

No todo ataque o error de seguridad se manifiesta en el sistema de archivos de
un servidor, sino que pueden residir únicamente en la memoria principal del
mismo. A su vez, existen escenarios donde estos dos niveles de memoria no se
encuentren en sincronía, haciendo que sea necesario poder acceder a ambos
durante una investigación. MASCHE, por las siglas de Memory Analysis Suite for
Checking the Harmony of Endpoints, es una respuesta a esto, desarrollada con el
fin de ser integrado a MIG.\\

El análisis forense de memoria es un área muy amplia y compleja dentro de la
respuesta a incidentes de seguridad informática, por lo que para este proyecto
nos enfocamos en los siguientes casos de uso, de manera de terminarlo en un
lapso de tiempo razonable:

\begin{enumerate}

\item Identificar procesos utilizando versiones vulnerables de librerías
    dinámicas.

\item Encontrar los procesos que tienen secretos filtrados, como claves
    privadas hechas publicas en algún incidente, en su memoria principal.

\item Localizar procesos con patrones de código que se sepan susceptibles a
    distintas vulnerabilidades. Notar que esto es necesario cuando una librería
    vulnerable es linkeada estáticamente, y no únicamente para código el
    programa en si mismo.

\item Proveer una API unificada de acceso a memoria principal de procesos en
    Windows NT, GNU/Linux y OS X, que permita agregar nuevas funcionalidades a
    MIG de manera práctica.

\end{enumerate}
