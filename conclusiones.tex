\section{Conclusiones}

El desarrollo de este trabajo sirvió para adentrarse en nuevas plataformas,
distintas a las vistas en el curso y a los demás del área de sistemas, y poder
encontrar similitudes y diferencias entre estas.\\

Una de las cosas que vale la pena destacar, es el trabajo que esto implica,
ya que alejarse del espacio de confort a la hora de trabajar conlleva no solo
tiempo dedicado a investigación, sino a comprender y/o aceptar diferencias y
limitaciones que en algunos casos parecen arbitrarias.\\

También es notable cómo los tres sistemas luego de una larga evolución en el
tiempo convergieron a una arquitectura similar. Si bien utilizan distintas
abstracciones para las mismas cosas, y manejan algunos asuntos de manera
particular, el manejo de memoria de los mismos, lo que a MASCHE más concierne,
se lleva a cabo de una manera similar. Creemos que esto se debe a que el diseño
de los sistemas operativos se ve fuertemente influenciado por el hardware
disponible en cada época, y por distintos requerimientos de performance y de
compatibilidad por motivos comerciales.\\

Otra cuestión destacable que podemos concluir es que el desarrollo de una
plataforma y las funcionalidades que la misma ofrece se encuentran atados a la
cultura de la comunidad que lo lleva acabo y utiliza. Es notable como en Linux,
una comunidad donde se valora ser abierto y la transparencia para con el
usuario, se exporta la información que necesitábamos de forma tal que su
utilización sea trivial. Windows, a pesar de que Microsoft lo mantenga como un
proyecto de código cerrado, siempre ha sido fuertemente enfocado en proveer una
plataforma de desarrollo que permita a otras empresas ofrecer sus productos que
corran en este. Lo cual requiere proveer de soporte y funcionalidades
para los desarrolladores de software de primer nivel, que se puede ver en
la amplia API de WIN32 y en la excelente documentación disponible en la MSDN.
Por último, Apple, más enfocado en controlar la experiencia del usuario de modo
de asegurar la calidad de la misma, provee sólo herramientas de desarrollo de
más alto nivel (exceptuando el desarrollo de drivers) que permiten llevar a
cabo la mayoría de los proyectos sin problemas, pero en cuanto uno se aleja de
este modelo la situación se complica debido a la falta de soporte y
documentación oficial.\\

En definitiva, a pesar de estas diferencias muy interesantes didácticamente, se
lograron cumplir los objetivos planteados para MASCHE dentro del tiempo
estipulado, y el éste se encuentra pronto a ser utilizado en producción en la
flota de servidores de Mozilla.\\

