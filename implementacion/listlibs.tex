\subsection{Módulo Listlibs}

El último modulo de MASCHE, Listlibs, permite listar las librerías dinámicas
cargadas por un proceso. Es importante notar que estas no son necesariamente
las mismas que se pueden ver en el binario que está corriendo el mismo, sino
que nos enfocamos en obtener la información acerca de cuales están
efectivamente cargadas.\\

Listlibs exporta simplemente la función ListLoadedLibraries. Ésta retorna una
lista de paths a las distintas librerías cargadas, y su implementación difiere
completamente en cada plataforma.\\

\subsubsection{Implementación en Linux}

En el caso de Linux las librerías dinámicas son cargadas utilizando mmap, una
syscall que mapea el contenido de un archivo en la memoria virtual del proceso.
Gracias a esto basta con analizar el contenido de /proc/[pid]/map para obtener
la información buscada.\\

La función ListLoadedLibraries en Linux simplemente parsea linea por linea el
archivo mencionado, y en caso de contar con un path a un archivo mapeado, éste
se agrega a la lista a devolver.\\

\subsubsection{Implementación en Windows}

Si bien no se cuenta con un sistema como procfs simplificando las cosas en
Windows, la API WIN32 es sumamente completa y nos permitió obtener la lista de
librerías cargadas sin demasiado problema.\\

La función EnumProcessModulesEx realiza algo muy similar a lo que estábamos
buscando implementar, con la diferencia de que en lugar de retornar la ruta a
los archivos cargados, se retornan HMODULEs. Éstos últimos son handles a un
modulo cargado (representado con la dirección en memoria del mismo) que pueden
utilizarse en distintos llamados de WIN32. En particular, los utilizamos con
GetModuleFileNameEx para obtener el archivo que cada modulo representaba.\\

Fuera de esto, solo tuvimos que encargarnos de alocar los recursos necesarios
para llamar a estas funciones, lo cual no es trivial y se logró mediante
iteraciones de prueba y error, pero no dificultó demasiado la tarea.\\

\subsubsection{Implementación en OS X}

Contrario a las otras plataformas, esta tarea resultó sorprendentemente
engorrosa en OS X. Como era de esperarse, no se encuentra disponible
documentación oficial alguna sobre cómo realizar esto. Pero fue particularmente
llamativo no encontrar funcionalidad que permita hacer esto en el código del
kernel ni las herramientas open source liberadas por Apple.\\

Para empeorar aún más la situación, toda la documentación extraoficial
encontrada en la web al respecto se encontraba ampliamente desactualizada y
recurría a buscar patrones reconocibles dentro de toda la memoria del proceso
para obtener algún indicio, no del todo fiable, de qué librerías habían sido
cargadas. Si bien esto ultimo es aceptable en muchas situaciones, se encuentra
en conflicto directo con el objetivo de MIG de interferir lo menos posible la
performance del sistema analizado.\\

Sin embargo, no encontrar información al respecto no puede ser un sinónimo de
que esta funcionalidad no pueda implementarse de una forma eficiente, ya que es
necesario para el funcionamiento normal del sistema operativo, en particular
para el linker dinámico.\\

Afortunadamente, fuimos a parar con el llamado RPC \textit{task\_info}, que
provee distinta información acerca de un proceso a quien lo llame. En este
caso, nos interesó obtener información acerca del linkeo dinámico del proceso,
llamada TASK\_DYLD\_INFO, el cual retorna la siguiente estructura:

\begin{lstlisting}[language=C]
struct task_dyld_info {
	mach_vm_address_t	all_image_info_addr;
	mach_vm_size_t		all_image_info_size;
	integer_t 		all_image_info_format;
};
\end{lstlisting}

\noindent El primer campo de la misma es la dirección dentro de la memoria del
proceso de la estructura \textit{all\_image\_info}, que contiene información
sobre todos los binarios cargados por el linker dinámico. Para obtener lo
buscado nos queda entonces leer esta estructura desde la memoria del proceso.\\

El problema al hacer esto, es que el proceso inspeccionado puede ser de 32 o 64
bits, independientemente de cómo este compilado MASCHE, por lo que tenemos que
soportar ambos. Por suerte, Apple agrego en versiones recientes del struct
\textit{task\_dyld\_info} su tercer parámetro, que nos provee esta información.
Una vez obtenido esto, podemos leer una de las siguientes estructuras,
dependiendo cual corresponda:

\begin{lstlisting}[language=C]
struct user32_dyld_all_image_infos {
	uint32_t	version;
	uint32_t	infoArrayCount;
	user32_addr_t	infoArray;
	user32_addr_t	notification;
	bool		processDetachedFromSharedRegion;
	bool		libSystemInitialized;
	user32_addr_t	dyldImageLoadAddress;
	user32_addr_t	jitInfo;
	user32_addr_t	dyldVersion;
	user32_addr_t	errorMessage;
	user32_addr_t	terminationFlags;
	user32_addr_t	coreSymbolicationShmPage;
	user32_addr_t	systemOrderFlag;
	user32_size_t	uuidArrayCount;
	user32_addr_t	uuidArray;
	user32_addr_t	dyldAllImageInfosAddress;
};

struct user64_dyld_all_image_infos {
	uint32_t	version;
	uint32_t	infoArrayCount;
	user64_addr_t	infoArray;
	user64_addr_t	notification;
	bool		processDetachedFromSharedRegion;
	bool		libSystemInitialized;
	user64_addr_t	dyldImageLoadAddress;
	user64_addr_t	jitInfo;
	user64_addr_t	dyldVersion;
	user64_addr_t	errorMessage;
	user64_addr_t	terminationFlags;
	user64_addr_t	coreSymbolicationShmPage;
	user64_addr_t	systemOrderFlag;
	user64_size_t	uuidArrayCount;
	user64_addr_t	uuidArray;
	user64_addr_t	dyldAllImageInfosAddress;
};
\end{lstlisting}

\noindent Es importante notar que las mismas no están exportadas en ningún
header accesible al usuario de OS X, sino que pudimos conocer su definición
inspeccionando el código de XNU. Por este motivo no utilizamos las mismas en el
código de MASCHE, ya que podría incurrir en una violación del copyright de
Apple. En lugar de esto, se utilizaron offsets y aritmética de punteros manual
para leer lo restante.

Aclarado esto último, veamos qué nos interesa de estos structs. infoArray es la
posición inicial de un arreglo con información de cada una de las librerías, e
infoArrayCount el tamaño del mismo, por lo que solo necesitamos de estos
campos.\\

El arreglo que queremos leer esta compuesto por elementos de la siguiente
estructura según el modo en el qué este corriendo el proceso:

\begin{lstlisting}[language=C]
struct user32_dyld_image_info {
	user32_addr_t	imageLoadAddress;
	user32_addr_t	imageFilePath;
	user32_ulong_t	imageFileModDate;
};

struct user64_dyld_image_info {
	user64_addr_t	imageLoadAddress;
	user64_addr_t	imageFilePath;
	user64_ulong_t	imageFileModDate;
};
\end{lstlisting}

Finalmente imageFilePath es la dirección inicial del path de la librería
representado como un string de C. Por lo que sólo queda copiar esta cadena de
texto.\\

Vale aclarar que copiar la misma no es algo trivial, ya que se desconoce su
tamaño, y se encuentra en la memoria del otro proceso. Pero luego de haber
implementado Memaccess esto nos resultó familiar. Incluso podrían haberse
utilizado las primitivas exportadas por Memaccess, pero al haber sido
desarrolladas con el fin de exportarse a Go utilizarlas desde C resultaba muy
engorroso.\\

Una vez copiados los strings, estamos en condiciones de devolver un arreglo con
todos los paths, y la tarea queda completa.\\

