\subsection{Módulo Memaccess}

Memaccess, el modulo principal de MASCHE, exporta una misma interfaz para
acceder a la memoria de un proceso (representado por un Process del modulo
anterior) en cada una de las plataformas soportadas.\\

Para lograr esto hizo falta abstraerse de la estructuración y el manejo de la
memoria en cada sistema. Esto se llevo acabo planteando la memoria de un proceso
como secuencia ordenada de bloques contiguos de memoria.\\

Cada uno de estos bloques es representado por una estructura MemoryRegion, que
posee la dirección de inicio del mismo (en el address space del proceso) y su
tamaño. Para obtenerlos se debe utilizar la función NextReadableMemoryRegion,
que dado un Process y una dirección retornara el próximo bloque de memoria
legible a partir de la misma.\\

Una decisión que debimos tomar a la hora de hacer esto fue qué hacer cuando nos
encontramos con un bloque de memoria sin permiso de lectura. Las opciones que
contemplábamos eran las siguientes:

\begin{enumerate}

\item Ignorar este la memoria y buscar el próximo bloque legible.

\begin{itemize}

\item Ventajas: Es fácil de implementar, y no interrumpe ni compromete de forma
alguna al proceso inspeccionado.

\item Desventajas: Un atacante puede aprovechar esto para ocultar información
de MASCHE manteniendo la memoria sin permiso de lectura la mayor cantidad de
tiempo posible.

\end{itemize}

\item Cambiar los permisos, copiar la memoria y restaurar los permisos
originales.

\begin{itemize}

\item Ventajas: Permite leer toda la memoria.

\item Desventajas: El cambio de permisos se realiza mientras se está ejecutando
el proceso, lo que puede llevarse a que se deje al mismo con permisos distintos
a los que esperaba debido a esto. Si esto ocurre existe la posibilidad de que
el proceso no funcione como es esperado o incluso a que el sistema operativo
detenga su ejecución debido a un error de permisos.

\end{itemize}

\item Pausar el proceso, hacer lo mismo que en la opción anterior, y volver a
correr el proceso.

\begin{itemize}

\item Ventajas: Permite leer toda la memoria sin riesgos de comprometer la
ejecución del proceso inspeccionado.

\item Desventajas: Daña la performance del sistema, ya que detiene la ejecución
de los procesos inspeccionados.

\end{itemize}

\end{enumerate}

\noindent Como se menciono anteriormente, priorizamos los objetivos de MIG a la
hora de tomar este tipo de decisiones, y éste debe interferir lo mínimo posible
con el sistema que está analizando, por lo que se optó por la primer opción.
Decidir no frenar los procesos resolvió también qué hacer cuando el layout de
la memoria cambia mientras se la está analizando. MASCHE no hace nada en
especial al respecto más que asegurarse de poder seguir adelante con el resto
de su trabajo.\\

Una vez conocido el layout de la memoria mediante los MemoryRegion disponibles
uno puede leer la misma con la función CopyMemory, que recibe un Process, una
dirección de memoria dentro del mismo, y un buffer a llenarse con la memoria
del proceso a partir de la dirección dada.\\

Con esto ya bastaría para poder inspeccionar la memoria de un proceso, pero
sabiendo que el principal objetivo es buscar patrones conocidos en la misma
se implementaron dos funciones que facilitan dicha tarea. Las mismas están
basadas en la abstracción que explicamos anteriormente, por lo que tienen una sola
implementación (en Go) para todas las plataformas.\\

La primera de estas, WalkMemory, trabaja sobre un Process y un buffer de bytes.
Esta va llenando el buffer de manera secuencial con la memoria del proceso
(iniciando donde dejó la lectura anterior) y llamando a una función provista
por el usuario con el buffer y la dirección inicial de la memoria que contiene
luego de cada vez que se llena. La segunda función, SlidingWalkMemory, es
similar pero llena el buffer empezando por la posición del medio de la memoria
que se había leído en la iteración anterior.\\

Se explican a continuación las implementaciones de NextReadableMemoryRegion y
CopyMemory de cada plataforma.\\

\subsubsection{Implementación en Linux}

Es en la implementación de este modulo donde más se puede apreciar el poder de
procfs y la simpleza que conlleva tener una interfaz de archivos para acceder a
la información.\\

En este caso, NextReadableMemoryRegion simplemente lee una a una las lineas de
el archivo maps dentro de /proc/[pid], buscando el primer memory map que
contenga la dirección provista, uniéndolo a los que le siguen en caso de ser
consecutivos y legibles, y retornando la información obtenida.\\

Sorprendentemente, CopyMemory fue aun más sencillo de implementar, pues solo
hace falta abrir el archivo /proc/[pid]/mem y utilizar como offset de lectura
la dirección provista.\\

\subsubsection{Implementación en OS X}

Nuevamente se creo una interfaz en C común para Windows y OS X, la misma sólo
contiene las funciones get\_next\_readable\_region copy\_process\_memory, que
luego son llamadas desde Go en la implementación de las funciones descriptas.\\

En el caso de get\_next\_readable\_region el sistema operativo cuenta con el
llamado RPC mach\_vm\_region\_recurse, que realiza una tarea muy similar. La
principal diferencia es que la memoria en OS X se encuentra organizada de
manera distinta, de manera recursiva, donde un mapa de memoria puede estar
incluido como submapa de otro. Esto es utilizado normalmente para poder
compartir las copias en memoria librerías dinámicas entre distintos procesos de
manera sencilla, y no es algo que suela utilizarse más allá de esto.
Simplemente debe tenerse en cuenta que al pedir una region de memoria se puede
obtener en realidad un submapa, y en este caso nos introducimos en el mismo
para encontrar la región buscada. A su vez, igual que en el caso de Linux,
unimos las distintas regiones consecutivas en una sola.\\

Copiar memoria resultó más sencillo, ya que se provee la función
mach\_vm\_read\_overwrite, que copia el contenido de la memoria del proceso a
un buffer local, sin limitaciones impuestas por el alineamiento ni limites
entre páginas.

\subsubsection{Implementación en Windows}

La implementación en windows resultó asombrosamente similar. La principal
diferencia, además de que no se usa un sistema de RPC, es que la memoria no
está organizada de manera jerárquica, simplificando la implementación de
get\_next\_readable\_region.\\

Comparándola con la implementación de OS X, VirtualQueryEx cumple el papel de
mach\_vim\_region\_recurse, y ReadProcessMemory de mach\_vm\_read\_overwrite.\\
