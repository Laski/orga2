\subsection{Módulo Process}

El primer modulo a analizar es Process. El mismo provee una manera
uniforme de describir a un proceso en las distintas plataformas y reservar los
recursos necesarios para poder obtener información sobre estos.\\

La funcionalidad exportada es simplemente una interfaz Process, y funciones
para listar, abrir (reservar los recursos mencionados) y cerrar (liberarlos)
procesos.\\

\subsubsection{Implementación en Linux}

En el caso de Linux basta el $pid$ de un proceso para tener una descripción del
mismo. Abrir un Process en este caso simplemente revisa que tengamos permisos
para leer los archivos correspondientes a este en procfs, y luego solo retorna
el $pid$.\\

Luego, la información restante que puede pedirse sobre el proceso es sacada de
\textit{exe} y \textit{status} dentro del directorio de procfs del proceso.\\

Para listar los procesos disponibles basta con inspeccionar los directorios de
\textit{/proc}.

\subsubsection{Implementación en Windows y OS X}

A diferencia de Linux donde todo pudo ser implementado en Go, Windows y OS X
requirieron utilizar llamadas a APIs en C. Si bien estas pueden hacerse
directamente desde Go, una decisión que mantuvimos a lo largo de todo el
proyecto es no abusar de esto, y exportar nuestras propias funciones de C en
caso de que el código se vuelva muy engorroso.\\

En este caso creamos una interfaz común en C que tanto la implementación de
Windows como la de OS X implementan. Ésta simplemente define cómo se representa,
abre y cierra un proceso en cada sistema.\\

En el caso de OS X un proceso es representado por un \textit{task\_t} que es un
renombre de un \textit{mach\_port\_t}, un puerto a través del cual comunicarse
con el proceso y poder inspeccionar distintas características de este. Abrir un
proceso luego consiste simplemente llamar a \textit{task\_for\_pid} y cerrarlo
desalocar el task.

Para obtener información de un proceso y listar los mismos se utiliza en OS X
la API POSIX del sistema, haciendo que implementar esto en Go no sea demasiado
complejo. En particular las funciones \textit{proc\_pidpath} y
\textit{proc\_listpids} de la librería \textit{libproc} hacen justo lo
necesario.\\

La API WIN32 ya provee un tipo \textit{HANDLE} que es utilizado para describir
un proceso, y funciones para abrirlos, cerrarlos y enumerarlos. Obtener el path
de un proceso abierto requiere mas trabajo, pero puede obtenerse listando los
módulos cargados por el mismo e inspeccionando el primero, que siempre es el
ejecutable.\\

