\subsection{OS X}

OS X es sin dudas la plataforma más particular de las tres soportadas. Para
entender el por qué de su arquitectura y comprender su funcionamiento es
necesario hacer una breve reseña histórica.\\

El kernel de Mac OS, XNU fue desarrollado originalmente para el sistema
operativo NeXTSTEP de la empresa NeXT, luego adquirida por Apple, lo que llevó
a su inclusión en la versión 10 de Mac OS, llamada actualmente OS X.\\

A su vez, XNU está basado en el kernel Mach. Éste surgió como un proyecto de
investigación en Carnegie Mellon University sobre micro kernels, programación
distribuida y computación en paralela. Como tal, presenta una arquitectura muy
diferente respecto a los demás sistemas comerciales, basándose fuertemente en la
orientación a objetos y pasajes de mensajes.\\

Con la integración de este kernel a Mac OS surgen distintos cambios. Los más
importantes son que se abandona la arquitectura interna de microkernel, pero se
mantiene expuesto el sistema de Inter Process Comunication en el cual se
basaba, y se agrega una capa de abstracción POSIX basada en código de FreeBSD.
De este modo, el sistema expone distintas APIs muy diferentes dependiendo a que
parte del mismo se quiera acceder.\\

Durante el desarrollo del proyecto intentamos utilizar la API POSIX en cuanto
era posible, pero lamentablemente esta es muy incompleta, y no fue suficiente
en la mayoría de los casos. Por esto mismo hubo que recurrir al sistema de IPC
de Mach, el cual se encuentra notablemente indocumentado. Pero esto no nos impidió
seguir adelante ya que, afortunadamente, XNU es open source en casi su totalidad. \\

Es destacable mencionar que, a diferencia de en UNIX, la comunicación con el
kernel usando la API de Mach se hace a través de Remote Procedure Calls. Esto
se logra utilizando un lenguaje especial, Mach Interface Language, para
declarar los mensajes aceptados por cada modulo, y transformando los llamados a
funciones (que serian syscalls en otro sistema) a envío de mensajes mediante un
puerto.  Debido a la falta de documentación al respecto, y a la antigüedad de
la poca que se encuentra disponible, no explicaremos en este informe más sobre
el tema, pero se puede investigar al respecto buscando las referencias
presentes en la bibliografía.\\

