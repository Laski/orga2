\subsection{GNU/Linux}
No haremos aquí una descripción exhaustiva de las características de Linux ya que su funcionamiento es ampliamente conocido y estudiado en el ámbito de las Ciencias de la Computación. En cambio nos centraremos en los aspectos principales que permitieron el desarrollo de los objetivos del trabajo.

Una de las principales características de Linux (y, en general, de los sistemas basados en Unix) es el principio de que \textit{todo es un archivo}, lo que provee una interfaz común para acceder a recursos de entrada/salida. Siguiendo este principio Linux cuenta con \textit{process filesystem (procfs)}, un pseudo sistema de archivos que se utiliza para acceder a información de los procesos que están corriendo en el sistema. \textit{Procfs} se encuentra en el directorio /proc, y en /proc/PID/ se encuentra la información de un proceso según su PID.

Estos son los pseudo archivos que utilizamos en el trabajo:

\begin{itemize}
 \item /proc/[pid]/maps contiene las regiones de memoria mapeadas por el proceso y sus correspondientes permisos.
 \item /proc/[pid]/mem se usa para acceder a las páginas de la memoria del proceso.
 \item /proc/[pid]/status brinda información del proceso, por ejemplo, el pathname del ejecutable o el PID del proceso padre.
\end{itemize}
