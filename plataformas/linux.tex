\subsection{Linux}

No haremos aquí una descripción detallada de la arquitectura de Linux ya que, si bien abunda el
material al respecto, no fue necesario tener un conocimiento profundo de la misma para
poder llevar adelante este proyecto, gracias a las distintas abstracciones que
el sistema provee.\\

Entre las distintas formas de comunicarse con el kernel Linux que tiene un
proceso de userland, una que nos fue especialmente útil para la implementación
de MASCHE fue procfs. Éste es un sistema de archivos especial implementado
originalmente en UNIX V8, luego portado a SVR4 y Plan 9, para ser ésta ultima
implementación clonada luego por casi toda la familia UNIX. La finalidad del
mismo es presentar distinta información del sistema y los procesos que están
corriendo en forma de archivos y carpetas.\\

Queremos remarcar que /proc no es un directorio con archivos reales que están
en disco, sino que es un sistema de archivos especial que funciona como vía
acceso a información de las estructuras del kernel. Si fuese de otra manera el
acceso a esta información debería realizarse a través de syscalls y perdería el
estándar de interfaz de acceso a archivo.\\

La implementación de Linux de procfs contiene un sinfín de archivos con
información no solo de los procesos sino también del sistema en si, pero para
este proyecto bastó con utilizar los siguientes:\\

\begin{itemize}

\item /proc/[pid]/maps contiene las regiones de memoria mapeadas por el proceso
    y sus correspondientes permisos.

\item /proc/[pid]/mem se usa para acceder a las páginas de la memoria del
    proceso. Es un archivo con el contenido de la memoria del proceso, donde un
    offset dentro del mismo corresponde a una posición de memoria en el address
    space el proceso.

\item /proc/[pid]/status brinda información del proceso, por ejemplo, el
    pathname del ejecutable o el PID del proceso padre.

\end{itemize}
