\documentclass[a4paper,10pt]{article}

\usepackage[margin=1in]{geometry}   % Setea el margen manualmente, todos iguales.
\usepackage[spanish]{babel}         % {Con estos dos anda
\usepackage[utf8]{inputenc}         % todo lo que es tildes y ñ}
\usepackage{fancyhdr}           %{Estos dos son para
\usepackage{ulem}
\pagestyle{fancyplain}          % el header copado}
\usepackage{color}          % Con esto puedo hacer la matufia de poner en color blanco un texto para engañar al formato
\usepackage{graphicx}   % Para insertar gráficos
\usepackage{array}          % Para usar arrays
\usepackage{hyperref}       % Para que tenga links el índice

%\usepackage{datetime}  % Para agregar automáticamente fecha/hora de compilación y otras cosas

\lhead{Organización del Computador 2}   % {Con esto se usa el header copado. También está \chead para
\rhead{Trabajo Práctico Final}    % el centro y comandos para el pie de página, buscar fancyhdr}
\renewcommand{\footrulewidth}{0.4pt}
\lfoot{Facultad de Ciencias Exactas y Naturales}
\rfoot{Universidad de Buenos Aires}
%\rfoot{\textit{}}
\usepackage{amsfonts}   % para simbolos de reales, naturales, etc. se usa \mathbb{•} y la letra
\usepackage{amsmath}    % para \implies
%\usepackage{algorithm}
%\usepackage{algorithmic}
\usepackage{caratula}
%%%%%%%%%%%%%%%%%%%%%%%%%%%%%%%%%%%%%
%      COMANDOS ÚTILES USADOS       %
%%%%%%%%%%%%%%%%%%%%%%%%%%%%%%%%%%%%%

% \section{title}       Te hace un título ``importante'' en negrita, numerado. También está \subsection{title} y \subsubsection{title}.
% \begin{itemize}       Te hace viñetas.
%   \item esto es un item   Cambiar itemize por enumerate te hace una numeración.
% \end{itemize}

% \textbf{text}         Te hace el texto en negrita (bold).
% \underline{text}      Te subraya el texto.

% \textsuperscript{text}    Te hace ``superindices'' con texto. En teoría subscript debería funcionar, pero se puede usar guion bajo entre llaves
%               y signos peso para hacerlo como alternativa. Sino buscar.

% \begin{tabular}{cols}     Es para hacer tablas. Se pone una c por cada columna deseada dentro de cols (si es que se desea centrada, l para justificar a 
%   a & b & c       izquierda, r a la derecha). Si se separa por espacios la tabla no tendrá líneas divisorias. Si se separa por | en lugar de 
% \end{tabular}         espacios, aparecerá una línea. Con || dos, y así. Luego para los elementos de las filas se escriben y se separan con ampersand (&).
%               Finalmente, para las líneas horizontales, se usa \hline para una linea en toda la tabla y \cline{i - j} te hace la linea desde
%               la celda i hasta la j, arrancando en 1.
%               Si en la columna se pone p(width) podés escribir un párrafo en la celda. Para hacer un enter con \\ no funciona porque te hace un
%               enter en la fila. Para eso se usa el comando \newline.
  
% \textcolor{color predefinido en palabras}{text}

%%%%%%%%%%%%%%%%%%%%%%%%%%%%%%%%%%%%%
%    FIN COMANDOS ÚTILES USADOS     %
%%%%%%%%%%%%%%%%%%%%%%%%%%%%%%%%%%%%%

\newcommand{\Gather}[1]{\begin{gather*}#1\end{gather*}}
%\newcommand{\Def}[1]{\textbf{Definición: }#1}
%\newcommand{\Prop}[1]{\textbf{Propiedad: }#1}
%\newcommand{\Teo}[1]{\textbf{Teorema: }#1}
\newcommand{\Obs}[1]{\textbf{Observación: }#1}
%\newcommand{\Amat}{A \in \mathbb{R}^{n\textnormal{x}n}}
\newcommand{\filtro}[1]{\textbf{\textit{#1}}}

\begin{document}

%%%%%%%%%%%%%%%%%%%%%%%%%%%
%           INICIO DE CARÁTULA          %
%%%%%%%%%%%%%%%%%%%%%%%%%%%

%% **************************************************************************
%
%  Package 'caratula', version 0.2 (para componer caratulas de TPs del DC).
%
%  En caso de dudas, problemas o sugerencias sobre este package escribir a
%  Nico Rosner (nrosner arroba dc.uba.ar).
%
% **************************************************************************



% ----- Informacion sobre el package para el sistema -----------------------

\NeedsTeXFormat{LaTeX2e}
\ProvidesPackage{caratula}[2003/4/13 v0.1 Para componer caratulas de TPs del DC]


% ----- Imprimir un mensajito al procesar un .tex que use este package -----

\typeout{Cargando package 'caratula' v0.2 (21/4/2003)}


% ----- Algunas variables --------------------------------------------------

\let\Materia\relax
\let\Submateria\relax
\let\Titulo\relax
\let\Subtitulo\relax
\let\Grupo\relax


% ----- Comandos para que el usuario defina las variables ------------------

\def\materia#1{\def\Materia{#1}}
\def\submateria#1{\def\Submateria{#1}}
\def\titulo#1{\def\Titulo{#1}}
\def\subtitulo#1{\def\Subtitulo{#1}}
\def\grupo#1{\def\Grupo{#1}}


% ----- Token list para los integrantes ------------------------------------

\newtoks\intlist\intlist={}


% ----- Comando para que el usuario agregue integrantes

\def\integrante#1#2#3{\intlist=\expandafter{\the\intlist
	\rule{0pt}{1.2em}#1&#2&\tt #3\\[0.2em]}}


% ----- Macro para generar la tabla de integrantes -------------------------

\def\tablaints{%
	\begin{tabular}{|l@{\hspace{4ex}}c@{\hspace{4ex}}l|}
		\hline
		\rule{0pt}{1.2em}Integrante & LU & Correo electr\'onico\\[0.2em]
		\hline
		\the\intlist
		\hline
	\end{tabular}}


% ----- Codigo para manejo de errores --------------------------------------

\def\se{\let\ifsetuperror\iftrue}
\def\ifsetuperror{%
	\let\ifsetuperror\iffalse
	\ifx\Materia\relax\se\errhelp={Te olvidaste de proveer una \materia{}.}\fi
	\ifx\Titulo\relax\se\errhelp={Te olvidaste de proveer un \titulo{}.}\fi
	\edef\mlist{\the\intlist}\ifx\mlist\empty\se%
	\errhelp={Tenes que proveer al menos un \integrante{nombre}{lu}{email}.}\fi
	\expandafter\ifsetuperror}


% ----- Reemplazamos el comando \maketitle de LaTeX con el nuestro ---------

\def\maketitle{%
	\ifsetuperror\errmessage{Faltan datos de la caratula! Ingresar 'h' para mas informacion.}\fi
	\thispagestyle{empty}
	\begin{center}
	\vspace*{\stretch{2}}
	{\LARGE\textbf{\Materia}}\\[1em]
	\ifx\Submateria\relax\else{\Large \Submateria}\\[0.5em]\fi
	\par\vspace{\stretch{1}}
	{\large Departamento de Computaci\'on}\\[0.5em]
	{\large Facultad de Ciencias Exactas y Naturales}\\[0.5em]
	{\large Universidad de Buenos Aires}
	\par\vspace{\stretch{3}}
	{\Large \textbf{\Titulo}}\\[0.8em]
	{\Large \Subtitulo}
	\par\vspace{\stretch{3}}
	\ifx\Grupo\relax\else\textbf{\Grupo}\par\bigskip\fi
	\tablaints
	\end{center}
	\vspace*{\stretch{3}}
	\newpage}

\materia{Organización del Computador 2}
\titulo{Trabajo Práctico Final}

\grupo{Grupo}
\integrante{Martínez Suñé, Agustín}{630/11}{zerolink92@gmail.com}
\integrante{Lascano, Nahuel}{476/11}{laski.nahuel@gmail.com}
\integrante{Vanotti, Marco}{XXX/XX}{marcovanotti15@gmail.com}
\integrante{Palladino, Patricio}{XXX/XX}{email@patriciopalladino.com}

\begin{titlepage}
\maketitle
\thispagestyle{empty}
\end{titlepage}

%%%%%%%%%%%%%%%%%%%%%%%%%%%
%               FIN DE CARÁTULA         %
%%%%%%%%%%%%%%%%%%%%%%%%%%%

\tableofcontents
\clearpage

%%%%%%%%%%%%%%%%%%%%%%%%%%%
%                   DESARROLLO              %
%%%%%%%%%%%%%%%%%%%%%%%%%%%
 
\section{Introducción}

El siguiente informe se presenta como complemento y documentación del trabájo
practico final de la materia Organización del computador II.\\

El mismo fue realizado dentro del programa Mozilla Winter Of Security
2014\footnote{https://wiki.mozilla.org/Security/Automation/WinterOfSecurity2014}.
Éste tiene como finalidad introducir a alumnos de carreras afines a la
computación al mundo de la seguridad informatica y la automatización. Para
esto, el equipo de Security
Automation\footnote{https://wiki.mozilla.org/Security/Automation} de Mozilla
propuso una serie de proyectos para que distintos grupos de alumnos pudieran
encarar, contando con un tutor provisto por ellos.\\

En este trabajo se llevó a cabo uno de dichos proyectos, que concistió en
proveer de funcionalidades de inspección de memoria a la plataforma de análisis
forense en tiempo real y respuesta a incidentes Mozilla
Investigator\footnote{http://mig.mozilla.org}, la cual será introducida a
continuación.\\

\subsection{Mozilla Investigator}

Mozilla Investigator surge en respuesta a los cambios que se ha vivido en la
infraestructura con el advenimiento del cloud computing. Mientras que
antes los distintos servidores y servicios permanecian estaticos a lo largo del
tiempo, sujetos a revisiones lentas y estrictas a la hora de ser alterados, hoy
en dia la virtualización permite un dinamismo y escala en la infraestructura
que solia ser impensable. Este nuevo modelo demostro ser adecuado en un sin
numero de situaciones, pero no viene sin traer sus nuevos desafios.\\

Uno de estos desafios, y del cual Mozilla Investigator (MIG) se encarga, es
como manejar la seguridad y respuesta a incidentes en escalas nunca antes
vistas de manera rapida y efectiva.\\

Para llevar a cabo esta tarea, MIG provee un sistema distribuido que facilita
realizar distintos tipos de chequeos en los servidores de Mozilla. Esto se
logra con una arquitectura distribuida, en la cual los distintos servidores
cuentan con un agente corriendo de manera constante el cual lee una cola de
queries a medida que esta se va llenando. Una vez que un
query es leido, el agente actua de manera acorde y envia su respuesta.
El funcionamiento del sistema es orquestado por MIG Scheduler, uno o
más servidores distinguidos, encargados de distribuir los queries a
los agentes y comunicar las respuestas al usuario. De este modo un investigador
de Mozilla puede conectarse al servidor de MIG y realizar consultas a miles de
servidores en simultaneo en cuestion de segundos, permitiendo acelerar la
reacción frente a incidentes de seguridad y agilizando las actividades
contidianas de security operations.\\

Si bien MIG es usado hoy en dia en producción, se encuentra aun bajo
desarrollo, principalmente con respecto a qué tipo de queries puede
responder un agente. En la actualidad esto está limitado a responder consultas
sobre la integridad del sistema de archivos, como chequear el nombre, contenido
y hash de los mismos; y consultar el estado del stack de red del sistema donde
está corriendo.\\

Existen proyectos para agregar soporte para distintos tipos de queries, como
consultar y manejar reglas de los distintos firewalls, crear, bloquear y
destruir cuentas de usuarios, analizar el trafico de la red, acceso a bajo
nivel de los distintos componentes fisicos del servidor, y muchos otros. En
particular, uno de estos modulos, que pronto se encontrará en producción, es el
desarrollo para este trabajo, el cual permite analizar la memoria principal de
los procesos que se encuentran corriendo en el sistema, y que se introducierá a
continuación.\\

\subsection{MASCHE}

No todo ataque o error de seguridad se manifiesta en el sistema de archivos de
un servidor, sino que pueden residir unicamente en la memoria principal del
mismo. A su vez, existen escenarios donde estos dos niveles de memoria no se
encuentren en sincronía, haciendo que sea necesario poder acceder a ambos
durante una investigación. MASCHE, por las siglas de Memory Analysis Suite for
Checking the Harmony of Endpoints, es una respuesta a esto, desarrollada con el
fin de ser integrado a MIG.\\

El analisis forense de memoria es un area muy amplia y compleja dentro de la
respuesta a incidentes de seguridad informatica, por lo que para este proyecto
nos enfocamos en los siguientes casos de uso, de manera de terminarlo en un
lapzo de tiempo razonable:

\begin{enumerate}

\item (...)

\end{enumerate}

\section{Motivación}
\section{Implementación}
\subsection{Implementación general}
\subsection{Windows}
\subsection{Linux}
\subsection{Mac OS}
\subsection{Apéndice: CGO}

\end{document}
