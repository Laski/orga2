\documentclass[a4paper,10pt]{article}

\usepackage[margin=1in]{geometry}   % Setea el margen manualmente, todos iguales.
\usepackage[spanish]{babel}         % {Con estos dos anda
\usepackage[utf8]{inputenc}         % todo lo que es tildes y ñ}
\usepackage{fancyhdr}           %{Estos dos son para
\usepackage{ulem}
\pagestyle{fancyplain}          % el header copado}
\usepackage{color}          % Con esto puedo hacer la matufia de poner en color blanco un texto para engañar al formato
\usepackage{graphicx}   % Para insertar gráficos
\usepackage{array}          % Para usar arrays
\usepackage{hyperref}       % Para que tenga links el índice

%\usepackage{datetime}  % Para agregar automáticamente fecha/hora de compilación y otras cosas

\lhead{Organización del Computador 2}   % {Con esto se usa el header copado. También está \chead para
\rhead{Trabajo Práctico Final}    % el centro y comandos para el pie de página, buscar fancyhdr}
\renewcommand{\footrulewidth}{0.4pt}
\lfoot{Facultad de Ciencias Exactas y Naturales}
\rfoot{Universidad de Buenos Aires}
%\rfoot{\textit{}}
\usepackage{amsfonts}   % para simbolos de reales, naturales, etc. se usa \mathbb{•} y la letra
\usepackage{amsmath}    % para \implies
%\usepackage{algorithm}
%\usepackage{algorithmic}
\usepackage{caratula}
%%%%%%%%%%%%%%%%%%%%%%%%%%%%%%%%%%%%%
%      COMANDOS ÚTILES USADOS       %
%%%%%%%%%%%%%%%%%%%%%%%%%%%%%%%%%%%%%

% \section{title}       Te hace un título ``importante'' en negrita, numerado. También está \subsection{title} y \subsubsection{title}.
% \begin{itemize}       Te hace viñetas.
%   \item esto es un item   Cambiar itemize por enumerate te hace una numeración.
% \end{itemize}

% \textbf{text}         Te hace el texto en negrita (bold).
% \underline{text}      Te subraya el texto.

% \textsuperscript{text}    Te hace ``superindices'' con texto. En teoría subscript debería funcionar, pero se puede usar guion bajo entre llaves
%               y signos peso para hacerlo como alternativa. Sino buscar.

% \begin{tabular}{cols}     Es para hacer tablas. Se pone una c por cada columna deseada dentro de cols (si es que se desea centrada, l para justificar a
%   a & b & c       izquierda, r a la derecha). Si se separa por espacios la tabla no tendrá líneas divisorias. Si se separa por | en lugar de
% \end{tabular}         espacios, aparecerá una línea. Con || dos, y así. Luego para los elementos de las filas se escriben y se separan con ampersand (&).
%               Finalmente, para las líneas horizontales, se usa \hline para una linea en toda la tabla y \cline{i - j} te hace la linea desde
%               la celda i hasta la j, arrancando en 1.
%               Si en la columna se pone p(width) podés escribir un párrafo en la celda. Para hacer un enter con \\ no funciona porque te hace un
%               enter en la fila. Para eso se usa el comando \newline.

% \textcolor{color predefinido en palabras}{text}

%%%%%%%%%%%%%%%%%%%%%%%%%%%%%%%%%%%%%
%    FIN COMANDOS ÚTILES USADOS     %
%%%%%%%%%%%%%%%%%%%%%%%%%%%%%%%%%%%%%

\newcommand{\Gather}[1]{\begin{gather*}#1\end{gather*}}
%\newcommand{\Def}[1]{\textbf{Definición: }#1}
%\newcommand{\Prop}[1]{\textbf{Propiedad: }#1}
%\newcommand{\Teo}[1]{\textbf{Teorema: }#1}
\newcommand{\Obs}[1]{\textbf{Observación: }#1}
%\newcommand{\Amat}{A \in \mathbb{R}^{n\textnormal{x}n}}
\newcommand{\filtro}[1]{\textbf{\textit{#1}}}

\begin{document}

%%%%%%%%%%%%%%%%%%%%%%%%%%%
%           INICIO DE CARÁTULA          %
%%%%%%%%%%%%%%%%%%%%%%%%%%%

%% **************************************************************************
%
%  Package 'caratula', version 0.2 (para componer caratulas de TPs del DC).
%
%  En caso de dudas, problemas o sugerencias sobre este package escribir a
%  Nico Rosner (nrosner arroba dc.uba.ar).
%
% **************************************************************************



% ----- Informacion sobre el package para el sistema -----------------------

\NeedsTeXFormat{LaTeX2e}
\ProvidesPackage{caratula}[2003/4/13 v0.1 Para componer caratulas de TPs del DC]


% ----- Imprimir un mensajito al procesar un .tex que use este package -----

\typeout{Cargando package 'caratula' v0.2 (21/4/2003)}


% ----- Algunas variables --------------------------------------------------

\let\Materia\relax
\let\Submateria\relax
\let\Titulo\relax
\let\Subtitulo\relax
\let\Grupo\relax


% ----- Comandos para que el usuario defina las variables ------------------

\def\materia#1{\def\Materia{#1}}
\def\submateria#1{\def\Submateria{#1}}
\def\titulo#1{\def\Titulo{#1}}
\def\subtitulo#1{\def\Subtitulo{#1}}
\def\grupo#1{\def\Grupo{#1}}


% ----- Token list para los integrantes ------------------------------------

\newtoks\intlist\intlist={}


% ----- Comando para que el usuario agregue integrantes

\def\integrante#1#2#3{\intlist=\expandafter{\the\intlist
	\rule{0pt}{1.2em}#1&#2&\tt #3\\[0.2em]}}


% ----- Macro para generar la tabla de integrantes -------------------------

\def\tablaints{%
	\begin{tabular}{|l@{\hspace{4ex}}c@{\hspace{4ex}}l|}
		\hline
		\rule{0pt}{1.2em}Integrante & LU & Correo electr\'onico\\[0.2em]
		\hline
		\the\intlist
		\hline
	\end{tabular}}


% ----- Codigo para manejo de errores --------------------------------------

\def\se{\let\ifsetuperror\iftrue}
\def\ifsetuperror{%
	\let\ifsetuperror\iffalse
	\ifx\Materia\relax\se\errhelp={Te olvidaste de proveer una \materia{}.}\fi
	\ifx\Titulo\relax\se\errhelp={Te olvidaste de proveer un \titulo{}.}\fi
	\edef\mlist{\the\intlist}\ifx\mlist\empty\se%
	\errhelp={Tenes que proveer al menos un \integrante{nombre}{lu}{email}.}\fi
	\expandafter\ifsetuperror}


% ----- Reemplazamos el comando \maketitle de LaTeX con el nuestro ---------

\def\maketitle{%
	\ifsetuperror\errmessage{Faltan datos de la caratula! Ingresar 'h' para mas informacion.}\fi
	\thispagestyle{empty}
	\begin{center}
	\vspace*{\stretch{2}}
	{\LARGE\textbf{\Materia}}\\[1em]
	\ifx\Submateria\relax\else{\Large \Submateria}\\[0.5em]\fi
	\par\vspace{\stretch{1}}
	{\large Departamento de Computaci\'on}\\[0.5em]
	{\large Facultad de Ciencias Exactas y Naturales}\\[0.5em]
	{\large Universidad de Buenos Aires}
	\par\vspace{\stretch{3}}
	{\Large \textbf{\Titulo}}\\[0.8em]
	{\Large \Subtitulo}
	\par\vspace{\stretch{3}}
	\ifx\Grupo\relax\else\textbf{\Grupo}\par\bigskip\fi
	\tablaints
	\end{center}
	\vspace*{\stretch{3}}
	\newpage}

\materia{Organización del Computador 2}
\titulo{MASCHE\\Memory Analysis Suite for \\Checking the
Harmony of Endpoints}

\grupo{Grupo}
\integrante{Martínez Suñé, Agustín}{630/11}{zerolink92@gmail.com}
\integrante{Lascano, Nahuel}{476/11}{laski.nahuel@gmail.com}
\integrante{Vanotti, Marco}{229/10}{marcovanotti15@gmail.com}
\integrante{Palladino, Patricio}{218/10}{email@patriciopalladino.com}

\begin{titlepage}
\maketitle
\thispagestyle{empty}
\end{titlepage}

%%%%%%%%%%%%%%%%%%%%%%%%%%%
%               FIN DE CARÁTULA         %
%%%%%%%%%%%%%%%%%%%%%%%%%%%

\tableofcontents
\clearpage

%%%%%%%%%%%%%%%%%%%%%%%%%%%
%                   DESARROLLO              %
%%%%%%%%%%%%%%%%%%%%%%%%%%%

\section{Introducción}

El siguiente informe se presenta como complemento y documentación del trabájo
practico final de la materia Organización del computador II.\\

El mismo fue realizado dentro del programa Mozilla Winter Of Security
2014\footnote{https://wiki.mozilla.org/Security/Automation/WinterOfSecurity2014}.
Éste tiene como finalidad introducir a alumnos de carreras afines a la
computación al mundo de la seguridad informatica y la automatización. Para
esto, el equipo de Security
Automation\footnote{https://wiki.mozilla.org/Security/Automation} de Mozilla
propuso una serie de proyectos para que distintos grupos de alumnos pudieran
encarar, contando con un tutor provisto por ellos.\\

En este trabajo se llevó a cabo uno de dichos proyectos, que concistió en
proveer de funcionalidades de inspección de memoria a la plataforma de análisis
forense en tiempo real y respuesta a incidentes Mozilla
Investigator\footnote{http://mig.mozilla.org}, la cual será introducida a
continuación.\\

\subsection{Mozilla Investigator}

Mozilla Investigator surge en respuesta a los cambios que se ha vivido en la
infraestructura con el advenimiento del cloud computing. Mientras que
antes los distintos servidores y servicios permanecian estaticos a lo largo del
tiempo, sujetos a revisiones lentas y estrictas a la hora de ser alterados, hoy
en dia la virtualización permite un dinamismo y escala en la infraestructura
que solia ser impensable. Este nuevo modelo demostro ser adecuado en un sin
numero de situaciones, pero no viene sin traer sus nuevos desafios.\\

Uno de estos desafios, y del cual Mozilla Investigator (MIG) se encarga, es
como manejar la seguridad y respuesta a incidentes en escalas nunca antes
vistas de manera rapida y efectiva.\\

Para llevar a cabo esta tarea, MIG provee un sistema distribuido que facilita
realizar distintos tipos de chequeos en los servidores de Mozilla. Esto se
logra con una arquitectura distribuida, en la cual los distintos servidores
cuentan con un agente corriendo de manera constante el cual lee una cola de
queries a medida que esta se va llenando. Una vez que un
query es leido, el agente actua de manera acorde y envia su respuesta.
El funcionamiento del sistema es orquestado por MIG Scheduler, uno o
más servidores distinguidos, encargados de distribuir los queries a
los agentes y comunicar las respuestas al usuario. De este modo un investigador
de Mozilla puede conectarse al servidor de MIG y realizar consultas a miles de
servidores en simultaneo en cuestion de segundos, permitiendo acelerar la
reacción frente a incidentes de seguridad y agilizando las actividades
contidianas de security operations.\\

Si bien MIG es usado hoy en dia en producción, se encuentra aun bajo
desarrollo, principalmente con respecto a qué tipo de queries puede
responder un agente. En la actualidad esto está limitado a responder consultas
sobre la integridad del sistema de archivos, como chequear el nombre, contenido
y hash de los mismos; y consultar el estado del stack de red del sistema donde
está corriendo.\\

Existen proyectos para agregar soporte para distintos tipos de queries, como
consultar y manejar reglas de los distintos firewalls, crear, bloquear y
destruir cuentas de usuarios, analizar el trafico de la red, acceso a bajo
nivel de los distintos componentes fisicos del servidor, y muchos otros. En
particular, uno de estos modulos, que pronto se encontrará en producción, es el
desarrollo para este trabajo, el cual permite analizar la memoria principal de
los procesos que se encuentran corriendo en el sistema, y que se introducierá a
continuación.\\

\subsection{MASCHE}

No todo ataque o error de seguridad se manifiesta en el sistema de archivos de
un servidor, sino que pueden residir unicamente en la memoria principal del
mismo. A su vez, existen escenarios donde estos dos niveles de memoria no se
encuentren en sincronía, haciendo que sea necesario poder acceder a ambos
durante una investigación. MASCHE, por las siglas de Memory Analysis Suite for
Checking the Harmony of Endpoints, es una respuesta a esto, desarrollada con el
fin de ser integrado a MIG.\\

El analisis forense de memoria es un area muy amplia y compleja dentro de la
respuesta a incidentes de seguridad informatica, por lo que para este proyecto
nos enfocamos en los siguientes casos de uso, de manera de terminarlo en un
lapzo de tiempo razonable:

\begin{enumerate}

\item (...)

\end{enumerate}

\newpage

\section{Introducción a las plataformas soportadas}

Como se menciono anteriormente, uno de los objetivos del proyecto MASCHE es que
sea multiplataforma, soportando la familia NT de Microsoft Windows, GNU/Linux
en sus versiones mas recientes, y OS X.\\

A continuación se presenta una breve introducción a cada una de estas
plataformas, enfocándonos principalmente en los aspectos pertinentes al
proyecto, pero sin entrar en demasiado detalle, ya que explicar el
funcionamiento de tres sistemas operativos escapa totalmente de lo que se
pretende hacer en este trabajo.\\


\subsection{Linux}

No haremos aquí una descripción detallada de la arquitectura de Linux ya que, si bien abunda el
material al respecto, no fue necesario tener un conocimiento profundo de la misma para
poder llevar adelante este proyecto, gracias a las distintas abstracciones que
el sistema provee.\\

Entre las distintas formas de comunicarse con el kernel Linux que tiene un
proceso de userland, una que nos fue especialmente útil para la implementación
de MASCHE fue $procfs$. Este es un sistema de archivos especial implementado
originalmente en UNIX V8, luego portado a SVR4 y Plan 9, para ser esta ultima
implementación clonada luego por casi toda la familia UNIX. La finalidad del
mismo es presentar cierta información del sistema y los procesos que están
corriendo en forma de archivos y carpetas.\\

Hace falta remarcar que $/proc$ no es un directorio con archivos reales que están
en disco sino que es un sistema de archivos especial que funciona como vía de
acceso a información de las estructuras del kernel. Sin él, el
acceso a esta información debería realizarse a través de syscalls y perdería el
estándar de interfaz de acceso a archivo.\\

La implementación de Linux de procfs contiene un sinfín de archivos con
información no solo de los procesos sino también del sistema en sí, pero para
este proyecto bastó con utilizar los siguientes:\\

\begin{itemize}

\item /proc/[pid]/maps contiene las regiones de memoria mapeadas por el proceso
    y sus correspondientes permisos.

\item /proc/[pid]/mem se usa para acceder a las páginas de la memoria del
    proceso. Es un archivo con el contenido de la memoria del proceso, donde un
    offset dentro del mismo corresponde a una posición de memoria en el address
    space el proceso.

\item /proc/[pid]/status brinda información del proceso, por ejemplo, el
    pathname del ejecutable o el PID del proceso padre.

\end{itemize}

\subsection{Windows NT}

\subsection{OS X}

OS X es sin dudas la plataforma más particular de las tres soportadas. Para
entender el por qué de su arquitectura y comprender su funcionamiento es
necesario hacer una breve reseña histórica.\\

El kernel de Mac OS, XNU fue desarrollado originalmente para el sistema
operativo NeXTSTEP de la empresa NeXT, luego adquirida por Apple, lo que llevó
a su inclusión en la versión 10 de Mac OS, llamada actualmente OS X.\\

A su vez, XNU está basado en el kernel Mach. Éste surgió como un proyecto de
investicación en Carnegie Mellon University sobre micro kernels, programacion
distribuida y computación en paralela. Cómo tal, presenta una arquitectura muy
diferente respecto a los demas sistemas comeriales, basandose fuertemente en la
orientación a objetos y pasajes de mensajes.\\

Con la integración de este kernel a Mac OS surgen distintos cambios. Los más
importantes son que se abandona la arquitectura interna de microkernel, pero se
mantiene expuesto el sistema de Inter Process Comunication en el cual se
basaba, y se agrega una capa de abstracción POSIX basada en código de FreeBSD.
De este modo, el sistema expone distintas APIs muy diferentes dependiendo a que
parte del mismo se quiera acceder.\\

Durante el desarrollo del proyecto se intento utilizar la API POSIX en cuanto
era posible, pero lamentablemente esta es muy incompleta, y no fue suficiente
en la mayoría de los casos. Por esto mismo hubo que recurrir al sistema de IPC
de Mach, el cual se encuentra notablemente indocumentado. Afortunadamente el el
este XNU es open source en casi su totalidad, por lo que ésto no nos impidio
seguir adelante.\\

Es destacable mencionar que, a diferencia de en UNIX, la comunicación con el
kernel usando la API de Mach se hace a traves de Remote Procedure Calls. Esto
se logra utilizando un lenguaje especial, Mach Interface Language, para
declarar los mensajes aceptados por cada modulo, y transformando los llamados a
funciones (que serian syscalls en otro sistema) a envio de mensajes mediante un
puerto.  Debido a la falta de documentación al respecto, y a la antiguedad de
la poca que se encuentra disponible, no explicaremos en este informe más sobre
el tema, pero se puede investigar al respecto buscando las referencias
presentes en la bibliografía.\\


\newpage

\section{Implementación de MASCHE}

En esta sección se explica cómo esta implementado MASCHE, los módulos que lo
conforman y su interacción con los distintos sistemas operativos.\\

Siendo MASCHE parte del proyecto Mozilla Investigator, distintas decisiones de
diseño se vieron atadas a las de éste último. Por empezar, el mismo se
encuentra desarrollado en Go\footnote{Para una breve introducción a Go ver el
primer apéndice.}, por ser el lenguaje en el que MIG está implementado. A su
vez, MASCHE no pretende ser una aplicación en si misma, sino que fue
desarrollado como un conjunto de módulos de Go para ser utilizados en MIG con
el fin de agregar soporte para nuevas funcionalidades en el mismo. Por ultimo,
al momento de tomar decisiones que implicaron algún tipo de \textit{trade-off}
se priorizaron los objetivos de MIG aun cuando esto no fuese lo ideal para
MASCHE como proyecto independiente.\\

Se describen a continuación cada uno de los módulos del proyecto.\\

\subsection{Módulo Process}

El primer modulo a ser a analizar es Process. El mismo provee una manera
uniforme de describir a un proceso en las distintas plataformas, y alocar los
recursos necesarios para poder obtener información sobre estos.\\

La funcionalidad exportada es simplemente una interfaz Process, y funciones
para listar, abrir (alocar los recursos mencionados) y cerrar (desalocarlos)
procesos.\\

\subsubsection{Implementación en Linux}

En el caso de Linux basta el pid de un proceso para tener una descripción del
mismo. Abrir un Process en este caso simplemente chequea que tengamos permisos
para leer los archivos correspondientes a este en procfs, y luego solo retorna
el pid.\\

Luego, la información restante que puede pedirse sobre el proceso es sacada de
\textit{exe} y \textit{status} dentro del directorio de procfs del proceso.\\

Para listar los procesos disponibles basta con inspeccionar los directorios de
\textit{/proc}.

\subsubsection{Implementación en Windows y OS X}

A diferencia de en Linux donde todo pudo ser implementado en Go, Windows y OS X
requirieron utilizar llamadas a APIs en C. Si bien estas pueden hacerse
directamente desde Go, una decisión que mantuvimos a lo largo de todo el
proyecto es no abusar de esto, y exportar nuestras propias funciones de C en
caso de que el código se vuelva muy engorroso.\\

En este caso creamos una interfaz común en C que tanto la implementación de
Windows como la de OS X implementan. Ésta simplemente define cómo se representa,
abre y cierra un proceso en cada sistema.\\

En el caso de OS X un proceso es representado por un \textit{task\_t} que es un
renombre de un \textit{mach\_port\_t}, un puerto a través del cual comunicarse
con el proceso y poder inspeccionar distintas características de este. Abrir un
proceso luego consiste simplemente llamar a \textit{task\_for\_pid} y cerrarlo
desalocar el task.

Para obtener información de un proceso y listar los mismos se utiliza en OS X
la API POSIX del sistema, haciendo que implementar esto en Go no sea demasiado
complejo. En particular las funciones \textit{proc\_pidpath} y
\textit{proc\_listpids} de la librería \textit{libproc} hacen justo lo
necesario.\\

La API WIN32 ya provee un tipo \textit{HANDLE} que es utilizado para describir
un proceso, y funciones para abrirlos, cerrarlos y enumerarlos. Obtener el path
de un proceso abierto requiere mas trabajo, pero puede obtenerse listando los
módulos cargados por el mismo e inspeccionando el primero, que siempre es el
ejecutable.\\


\subsection{Módulo Memaccess}

Memaccess, el modulo principal de MASCHE, exporta una misma interfaz para
acceder a la memoria de un proceso (representado por un Process del modulo
anterior) en cada una de las plataformas soportadas.\\

Para lograr esto hizo falta abstraerse de la estructuración y el manejo de la
memoria en cada sistema. Esto se llevo acabo planteando la memoria de un proceso
como secuencia ordenada de bloques contiguos de memoria.\\

Cada uno de estos bloques es representado por una estructura MemoryRegion, que
posee la dirección de inicio del mismo (en el address space del proceso) y su
tamaño. Para obtenerlos se debe utilizar la función NextReadableMemoryRegion,
que dado un Process y una dirección retornara el próximo bloque de memoria
legible a partir de la misma.\\

Una decisión que debimos tomar a la hora de hacer esto fue qué hacer cuando nos
encontramos con un bloque de memoria sin permiso de lectura. Las opciones que
contemplábamos eran las siguientes:

\begin{enumerate}

\item Ignorar este la memoria y buscar el próximo bloque legible.

\begin{itemize}

\item Ventajas: Es fácil de implementar, y no interrumpe ni compromete de forma
alguna al proceso inspeccionado.

\item Desventajas: Un atacante puede aprovechar esto para ocultar información
de MASCHE manteniendo la memoria sin permiso de lectura la mayor cantidad de
tiempo posible.

\end{itemize}

\item Cambiar los permisos, copiar la memoria y restaurar los permisos
originales.

\begin{itemize}

\item Ventajas: Permite leer toda la memoria.

\item Desventajas: El cambio de permisos se realiza mientras se está ejecutando
el proceso, lo que puede llevarse a que se deje al mismo con permisos distintos
a los que esperaba debido a esto. Si esto ocurre existe la posibilidad de que
el proceso no funcione como es esperado o incluso a que el sistema operativo
detenga su ejecución debido a un error de permisos.

\end{itemize}

\item Pausar el proceso, hacer lo mismo que en la opción anterior, y volver a
correr el proceso.

\begin{itemize}

\item Ventajas: Permite leer toda la memoria sin riesgos de comprometer la
ejecución del proceso inspeccionado.

\item Desventajas: Daña la performance del sistema, ya que detiene la ejecución
de los procesos inspeccionados.

\end{itemize}

\end{enumerate}

\noindent Como se menciono anteriormente, priorizamos los objetivos de MIG a la
hora de tomar este tipo de decisiones, y éste debe interferir lo mínimo posible
con el sistema que está analizando, por lo que se optó por la primer opción.
Decidir no frenar los procesos resolvió también qué hacer cuando el layout de
la memoria cambia mientras se la está analizando. MASCHE no hace nada en
especial al respecto más que asegurarse de poder seguir adelante con el resto
de su trabajo.\\

Una vez conocido el layout de la memoria mediante los MemoryRegion disponibles
uno puede leer la misma con la función CopyMemory, que recibe un Process, una
dirección de memoria dentro del mismo, y un buffer a llenarse con la memoria
del proceso a partir de la dirección dada.\\

Con esto ya bastaría para poder inspeccionar la memoria de un proceso, pero
sabiendo que el principal objetivo es buscar patrones conocidos en la misma
se implementaron dos funciones qué facilitan dicha tarea. Las mismas están
basadas en la funcionalidad que ya se explicó, por lo que tienen una sola
implementación para todas las plataformas.\\

La primera de estas, WalkMemory, trabaja sobre un Process y un buffer de bytes.
Ella va llenando el buffer de manera secuencial con la memoria del proceso
(iniciando donde dejó la lectura anterior) y llamando a una función provista
por el usuario con el buffer y la dirección inicial de la memoria que contiene
luego de cada vez que se llena. La segunda función, SlidingWalkMemory, es
similar pero llena el buffer empezando por la posición del medio de la memoria
que se había leído en la iteración anterior.\\

Se explican a continuación las implementaciones de NextReadableMemoryRegion y
CopyMemory de cada plataforma.\\

\subsubsection{Implementación en Linux}

Es en la implementación de este modulo donde más se puede apreciar el poder de
procfs y la simpleza que conlleva tener una interfaz de archivos para acceder a
la información.\\

En este caso, NextReadableMemoryRegion simplemente lee una a una las lineas de
el archivo maps dentro de /proc/[pid], buscando el primer memory map que
contenga la dirección provista, uniéndolo a los que le siguen en caso de ser
consecutivos y legibles, y retornando la información obtenida.\\

Sorprendentemente, CopyMemory fue aun más sencillo de implementar, pues solo
hace falta abrir el archivo /proc/[pid]/mem y utilizar como offset de lectura
la dirección provista.\\

\subsubsection{Implementación en OS X}

Nuevamente se creo una interfaz en C común para Windows y OS X, la misma sólo
contiene las funciones get\_next\_readable\_region copy\_process\_memory, que
luego son llamadas desde Go en la implementación de las funciones descriptas.\\

En el caso de get\_next\_readable\_region el sistema operativo cuenta con el
llamado RPC mach\_vm\_region\_recurse, que realiza una tarea muy similar. La
principal diferencia es que la memoria en OS X se encuentra organizada de
manera distinta, de manera recursiva, donde un mapa de memoria puede estar
incluido como submapa de otro. Esto es utilizado normalmente para poder
compartir las copias en memoria librerías dinámicas entre distintos procesos de
manera sencilla, y no es algo que suela utilizarse más allá de esto.
Simplemente debe tenerse en cuenta que al pedir una region de memoria se puede
obtener en realidad un submapa, y en este caso nos introducimos en el mismo
para encontrar la región buscada. A su vez, igual que en el caso de Linux,
unimos las distintas regiones consecutivas en una sola.\\

Copiar memoria resultó más sencillo, ya que se provee la función
mach\_vm\_read\_overwrite, que copia el contenido de la memoria del proceso a
un buffer local, sin limitaciones impuestas por el alineamiento ni limites
entre páginas.

\subsubsection{Implementación en Windows}

La implementación en windows resultó asombrosamente similar. La principal
diferencia, además de que no se usa un sistema de RPC, es que la memoria no
está organizada de manera jerárquica, simplificando la implementación de
get\_next\_readable\_region.\\

Comparándola con la implementación de OS X, VirtualQueryEx cumple el papel de
mach\_vim\_region\_recurse, y ReadProcessMemory de mach\_vm\_read\_overwrite.\\

\subsection{Módulo Memsearch}

Memsearch se basa exclusivamente en la funcionalidad provista por Memaccess,
por lo que basto una única implementación en Go para todas las plataformas.
Este modulo permite la búsqueda de distintos patrones en la memoria del
proceso.\\

El modulo exporta únicamente las siguientes dos funciones:

\begin{itemize}

\item FindBytesSequence: Busca la primer aparición literal de una cadena de
bytes a partir de una dirección dada en la memoria de un proceso. Su
implementación consiste en llamar a memaccess.SlidingWalkMemory pasándole una
función que busque de manera secuencial en cada buffer leído.\\

\item FindRegexpMatch: Es similar a la anterior, pero en lugar de buscar
ocurrencias literales utiliza una regexp para encontrar secciones de memoria
que matcheen con esta. La implementación también es parecida, reemplazándose la
búsqueda secuencial por su equivalente en expresiones regulares.\\

\end{itemize}


\subsection{Módulo Listlibs}

\newpage

\section{Apéndices}

\subsection{Introducción a Go}

Go es un lenguaje de programación de codigo abierto diseñado y desarrollado por
Robert Griesemer, Rob Pike, y Ken Thompson en Google en el año 2007. Según su
sitio web oficial\footnote{https://golang.org/ref/spec\#Introduction} es un
lenguaje de propósito general diseñado ``pensando en programación de
sistemas''. Tiene una sintáxis similar a C, es estáticamente tipado, posee
garbage collection y soporta concurrencia de manera primitiva.\\

Estos son algunos aspectos importantes para entender el lenguaje:\\

\begin{itemize}

\item \textbf{Paquetes:} los programas se organizan en paquetes, todo programa
     empieza su ejecución en el paquete \textit{main}.

\item \textbf{Punteros:} Go no tiene aritmética de punteros y la indirección
     de un puntero es transparente al programador.

\item \textbf{Slices:} Una abstracción de secuencias de datos del mismo tipo.
     Es deseable usar este tipo de construcciones en lugar de directamente
     arreglos.

\item \textbf{error:} Tipo de datos utilizado para el manejo de errores. Go
     está pensado para manejar errores explícitamente donde ocurren, por lo que
     es deseable que cada función devuelva, además de su resultado, un valor de
     \textit{error}.

\item \textbf{Go-rutina:} Una función ejecutándose concurrentemente con otras
     go-rutinas.

\item \textbf{Channels:} Mecanismo de comunicación que permite el envío y
     recepción de datos entre dos go-rutinas.

\end{itemize}

Así se escribe un programa en Go:

\begin{verbatim}
package main

import (
    "fmt"
    "time"
)

func say(s string) {
    for i := 0; i < 5; i++ {
        time.Sleep(100 * time.Millisecond)
        fmt.Println(s)
    }
}

func main() {
    go say("mundo")
    say("hola")
}
\end{verbatim}

El programa es un ``Hola mundo'' concurrente: imprime cinco veces ``hola''
mientras imprime cinco veces ``mundo'' desde otro hilo de ejecución.

\subsection{CGO: invocando codigo C desde GO}

Go provee una forma nativa de llamar a codigo C desde un programa escrito en
Go\footnote{http://golang.org/cmd/cgo/}. Esto fue necesario en nuestro proyecto
para las implentaciones en Mac OSX y en Windows ya que una buena parte de las
funcionalidades sólo podía ser escrita en un lenguaje de bajo nivel como C.  El
pseudo-paquete \textit{C} es el encargado de la comunicacion con codigo C, al
importar este paquete se puede especificar el header para la compilación en C:

\begin{verbatim}
// #include <stdio.h>
// #include <errno.h>
import "C"
\end{verbatim}

Así, a lo largo del codigo en Go podemos, a través de este paquete, invocar
funciones en C:

\begin{verbatim}
package main

// typedef int (*intFunc) ();
//
// int
// bridge_int_func(intFunc f)
// {
//		return f();
// }
//
// int fortytwo()
// {
//	    return 42;
// }
import "C"
import "fmt"

func main() {
	f := C.intFunc(C.fortytwo)
	fmt.Println(int(C.bridge_int_func(f)))
	// Output: 42
}
\end{verbatim}

En contrapartida funciones de Go pueden ser exportadas para utilizarlas en
codigo C de la siguiente manera:

\begin{verbatim}
//export MyFunction
func MyFunction(arg1, arg2 int, arg3 string) int64 {...}

//export MyFunction2
func MyFunction2(arg1, arg2 int, arg3 string) (int64, *C.char) {...}
\end{verbatim}

\subsection{Ejemplo de uso: Detectando Heartbleed}

El 2014 fue un año sumamente activo en materia de seguridad informática, con
vulnerabilidades e incidentes de escala nunca antes vistos. Algunos ejemplos de
estos son la filtración de contenido adulto perteneciente a distintas
celebridades proveniente de iCloud, que llevo a un cambio radical en la postura
de Apple respecto a la privacidad de sus usuarios; el hackeo a Sony, que
publico numerosas películas antes de su estreno y libero una gran cantidad de
información sensible; POODLE, que puso fin a la historia de SSLv3; goto fail,
poniendo en riesgo a todos los usuarios de productos de Apple; Shellshock, que
permitía la ejecución remota de código en infinidad de servidores Linux. Pero
sin dudas el ejemplo más anecdótico y dañino de estos es la vulnerabilidad
conocida como Heartbleed (CVE-2014-0160 acorde a la base de datos MITRE).\\

Ésta se debió un error de manejo de memoria en la implementación de la extensión
de heartbeat de TLS provista por la librería OpenSSL. Éste error causaba que un
atacante pudiera leer toda la memoria de un sistema vulnerable sin dejar log
alguno al respecto.\\

Este problema se encontraba en las versiones 1.0.1 a 1.0.1f, y 1.0.2-beta, por
lo que el bug llevaba dos años en sistemas en producción cuando fue
descubierto. Debido a esto, las versiones vulnerables de OpenSSL se encontraban
en millones de servidores de todo el mundo. Como suele ser usual, un parche
solucionando el problema salió casi en simultaneo con el anuncio del mismo,
pero esto implicó recompilar software y/o actualizar OpenSSL en infinidad de
lugares, haciendo difícil coordinar la tarea de modo de asegurarse que se haga
eficientemente y no se dejen servicios vulnerables por error.\\

Esto es un escenario donde contar con la ayuda de MIG, y en particular de
MASCHE, podría haber simplificado mucho la mitigación del problema, y aumentado
la confianza que el equipo de seguridad tiene en la ejecución de la misma.\\

Afortunadamente OpenSSL es utilizado normalmente cómo una librería dinámica, por
lo que utilizando los módulos Process y Listlibs se podrá encontrar que procesos
deben reiniciarse luego de actualizar esta librería.\\

Pero esto no alcanza para estar seguros de que ningún proceso vulnerable está
corriendo en el sistema, ya que la librería con el error puede haber sido
linkeada estáticamente. Para encontrar este tipo de procesos se utiliza
Memaccess, creando una firma de la vulnerabilidad y utilizando MASCHE para
chequear si esta presente o no en cada uno de los procesos.\\

En el caso de Heartbleed, podemos crear una firma viendo cómo se soluciono el
bug, y detectando si esta presente este arreglo o no. Para hacerlo,
inspeccionamos el commit que lo arreglo, en particular nos enfocamos en la
siguiente parte:

\begin{lstlisting}
diff --git a/ssl/t1_lib.c b/ssl/t1_lib.c
index a2e2475..bcb99b8 100644
--- a/ssl/t1_lib.c
+++ b/ssl/t1_lib.c
@@ -3969,16 +3969,20 @@ tls1_process_heartbeat(SSL *s)
 	unsigned int payload;
 	unsigned int padding = 16; /* Use minimum padding */

-	/* Read type and payload length first */
-	hbtype = *p++;
-	n2s(p, payload);
-	pl = p;
-
 	if (s->msg_callback)
 		s->msg_callback(0, s->version, TLS1_RT_HEARTBEAT,
 			&s->s3->rrec.data[0], s->s3->rrec.length,
 			s, s->msg_callback_arg);

+	/* Read type and payload length first */
+	if (1 + 2 + 16 > s->s3->rrec.length)
+		return 0; /* silently discard */
+	hbtype = *p++;
+	n2s(p, payload);
+	if (1 + 2 + payload + 16 > s->s3->rrec.length)
+		return 0; /* silently discard per RFC 6520 sec. 4 */
+	pl = p;
+
 	if (hbtype == TLS1_HB_REQUEST)
 		{
 		unsigned char *buffer, *bp;
\end{lstlisting}

Luego es cuestión de crear una firma de la función
\textit{tls1\_process\_heartbeat} en su vulnerable. Si bien esto no es algo
trivial, es un problema cotidiano en la práctica, y no nos embarcamos en éste
durante el proyecto. Pero a modo de ejemplo, generamos la firma para una
version con símbolos de OpenSSL compilada con GCC en su version más reciente de
Ubuntu 14.04. Para esto, luego de compilar la librería y linkearla
estáticamente a nginx (a modo de ejemplo), utilizamos gdb para encontrar el
código de la función:

\begin{lstlisting}
(gdb) disassemble /r tls1_process_heartbeat
41 56                           push   %r14
41 55                           push   %r13
41 54                           push   %r12
55                              push   %rbp
48 89 fd                        mov    %rdi,%rbp
53                              push   %rbx
48 83 ec 10                     sub    $0x10,%rsp
48 8b 97 80 00 00 00            mov    0x80(%rdi),%rdx
4c 8b a2 30 01 00 00            mov    0x130(%rdx),%r12
41 0f b6 5c 24 01               movzbl 0x1(%r12),%ebx
41 0f b6 44 24 02               movzbl 0x2(%r12),%eax
45 0f b6 2c 24                  movzbl (%r12),%r13d
c1 e3 08                        shl    $0x8,%ebx
09 c3                           or     %eax,%ebx
48 8b 87 98 00 00 00            mov    0x98(%rdi),%rax
48 85 c0                        test   %rax,%rax
74 23                           je     0x4b6252 <tls1_process_heartbeat+98>
44 8b 82 24 01 00 00            mov    0x124(%rdx),%r8d
48 8b 97 a0 00 00 00            mov    0xa0(%rdi),%rdx
49 89 f9                        mov    %rdi,%r9
4c 89 e1                        mov    %r12,%rcx
48 89 14 24                     mov    %rdx,(%rsp)
ba 18 00 00 00                  mov    $0x18,%edx
8b 37                           mov    (%rdi),%esi
31 ff                           xor    %edi,%edi
ff d0                           callq  *%rax
66 41 83 fd 01                  cmp    $0x1,%r13w
74 4f                           je     0x4b62a8 <tls1_process_heartbeat+184>
31 c0                           xor    %eax,%eax
66 41 83 fd 02                  cmp    $0x2,%r13w
74 0e                           je     0x4b6270 <tls1_process_heartbeat+128>
48 83 c4 10                     add    $0x10,%rsp
5b                              pop    %rbx
5d                              pop    %rbp
41 5c                           pop    %r12
41 5d                           pop    %r13
41 5e                           pop    %r14
c3                              retq
90                              nop
83 fb 12                        cmp    $0x12,%ebx
41 0f b6 54 24 03               movzbl 0x3(%r12),%edx
41 0f b6 4c 24 04               movzbl 0x4(%r12),%ecx
75 e1                           jne    0x4b6262 <tls1_process_heartbeat+114>
8b b5 a0 02 00 00               mov    0x2a0(%rbp),%esi
c1 e2 08                        shl    $0x8,%edx
09 ca                           or     %ecx,%edx
39 d6                           cmp    %edx,%esi
75 d2                           jne    0x4b6262 <tls1_process_heartbeat+114>
83 c6 01                        add    $0x1,%esi
c7 85 9c 02 00 00 00 00 00 00   movl   $0x0,0x29c(%rbp)
89 b5 a0 02 00 00               mov    %esi,0x2a0(%rbp)
eb bd                           jmp    0x4b6262 <tls1_process_heartbeat+114>
0f 1f 00                        nopl   (%rax)
44 8d 73 13                     lea    0x13(%rbx),%r14d
ba 14 0a 00 00                  mov    $0xa14,%edx
be 48 1a 6c 00                  mov    $0x6c1a48,%esi
44 89 f7                        mov    %r14d,%edi
e8 72 45 03 00                  callq  0x4ea830 <CRYPTO_malloc>
49 89 c5                        mov    %rax,%r13
c6 00 02                        movb   $0x2,(%rax)
89 d8                           mov    %ebx,%eax
49 8d 4d 03                     lea    0x3(%r13),%rcx
c1 e8 08                        shr    $0x8,%eax
49 8d 74 24 03                  lea    0x3(%r12),%rsi
48 89 da                        mov    %rbx,%rdx
41 88 45 01                     mov    %al,0x1(%r13)
41 88 5d 02                     mov    %bl,0x2(%r13)
48 89 cf                        mov    %rcx,%rdi
e8 eb 53 17 00                  callq  0x62b6d0 <memcpy>
48 8d 3c 18                     lea    (%rax,%rbx,1),%rdi
be 10 00 00 00                  mov    $0x10,%esi
e8 ed a6 05 00                  callq  0x5109e0 <RAND_pseudo_bytes>
44 89 f1                        mov    %r14d,%ecx
4c 89 ea                        mov    %r13,%rdx
be 18 00 00 00                  mov    $0x18,%esi
48 89 ef                        mov    %rbp,%rdi
e8 4a 7b 02 00                  callq  0x4dde50 <ssl3_write_bytes>
85 c0                           test   %eax,%eax
78 56                           js     0x4b6360 <tls1_process_heartbeat+368>
48 8b 85 98 00 00 00            mov    0x98(%rbp),%rax
48 85 c0                        test   %rax,%rax
74 2a                           je     0x4b6340 <tls1_process_heartbeat+336>
48 8b 95 a0 00 00 00            mov    0xa0(%rbp),%rdx
45 89 f0                        mov    %r14d,%r8d
49 89 e9                        mov    %rbp,%r9
41 81 e0 ff ff 01 00            and    $0x1ffff,%r8d
4c 89 e9                        mov    %r13,%rcx
bf 01 00 00 00                  mov    $0x1,%edi
48 89 14 24                     mov    %rdx,(%rsp)
ba 18 00 00 00                  mov    $0x18,%edx
8b 75 00                        mov    0x0(%rbp),%esi
ff d0                           callq  *%rax
4c 89 ef                        mov    %r13,%rdi
e8 48 48 03 00                  callq  0x4eab90 <CRYPTO_free>
48 83 c4 10                     add    $0x10,%rsp
31 c0                           xor    %eax,%eax
5b                              pop    %rbx
5d                              pop    %rbp
41 5c                           pop    %r12
41 5d                           pop    %r13
41 5e                           pop    %r14
c3                              retq
66 0f 1f 84 00 00 00 00 00      nopw   0x0(%rax,%rax,1)
4c 89 ef                        mov    %r13,%rdi
89 44 24 0c                     mov    %eax,0xc(%rsp)
e8 24 48 03 00                  callq  0x4eab90 <CRYPTO_free>
8b 44 24 0c                     mov    0xc(%rsp),%eax
e9 ed fe ff ff                  jmpq   0x4b6262 <tls1_process_heartbeat+114>
End of assembler dump.
\end{lstlisting}

Una vez obtenido esto, podemos utilizar memsearch.FindBytesSequence para
encontrar el código maquina arriba listado en hexadecimal dentro de cada
proceso. Es importante aclarar que esta firma dista de ser óptima, y está atada
a una sola version de OpenSSL y una sola configuración de compilación. Usar
memsearch.FindRegexpMatch puede ayudar a mejorar esto, pero la necesidad o no
de hacerlo dependerá de las políticas de configuration management de la
organización que esté utilizando MASCHE.\\

Se provee justo a este informe una maquina virtual que cuenta con versiones de
OpenSSL y nginx vulnerables a Heartbleed para recrear todo lo acá explicado.\\



\end{document}
