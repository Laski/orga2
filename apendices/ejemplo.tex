\subsection{Ejemplo de uso: Detectando Heartbleed}

El 2014 fue un año sumamente activo en materia de seguridad informática, con
vulnerabilidades e incidentes de escala nunca antes vistos. Algunos ejemplos de
estos son la filtración de contenido adulto perteneciente a distintas
celebridades proveniente de iCloud, que llevo a un cambio radical en la postura
de Apple respecto a la privacidad de sus usuarios; el hackeo a Sony, que
publico numerosas películas antes de su estreno y libero una gran cantidad de
información sensible; POODLE, que puso fin a la historia de SSLv3; goto fail,
poniendo en riesgo a todos los usuarios de productos de Apple; Shellshock, que
permitía la ejecución remota de código en infinidad de servidores Linux. Pero
sin dudas el ejemplo más anecdótico y dañino de estos es la vulnerabilidad
conocida como Heartbleed (CVE-2014-0160 acorde a la base de datos MITRE).\\

Ésta se debió un error de manejo de memoria en la implementación de la extensión
de heartbeat de TLS provista por la librería OpenSSL. Éste error causaba que un
atacante pudiera leer toda la memoria de un sistema vulnerable sin dejar log
alguno al respecto.\\

Este problema se encontraba en las versiones 1.0.1 a 1.0.1f, y 1.0.2-beta, por
lo que el bug llevaba dos años en sistemas en producción cuando fue
descubierto. Debido a esto, las versiones vulnerables de OpenSSL se encontraban
en millones de servidores de todo el mundo. Como suele ser usual, un parche
solucionando el problema salió casi en simultaneo con el anuncio del mismo,
pero esto implicó recompilar software y/o actualizar OpenSSL en infinidad de
lugares, haciendo difícil coordinar la tarea de modo de asegurarse que se haga
eficientemente y no se dejen servicios vulnerables por error.\\

Esto es un escenario donde contar con la ayuda de MIG, y en particular de
MASCHE, podría haber simplificado mucho la mitigación del problema, y aumentado
la confianza que el equipo de seguridad tiene en la ejecución de la misma.\\

Afortunadamente OpenSSL es utilizado normalmente cómo una librería dinámica, por
lo que utilizando los módulos Process y Listlibs se podrá encontrar que procesos
deben reiniciarse luego de actualizar esta librería.\\

Pero esto no alcanza para estar seguros de que ningún proceso vulnerable está
corriendo en el sistema, ya que la librería con el error puede haber sido
linkeada estáticamente. Para encontrar este tipo de procesos se utiliza
Memaccess, creando una firma de la vulnerabilidad y utilizando MASCHE para
chequear si esta presente o no en cada uno de los procesos.\\

En el caso de Heartbleed, podemos crear una firma viendo cómo se soluciono el
bug, y detectando si esta presente este arreglo o no. Para hacerlo,
inspeccionamos el commit que lo arreglo, en particular nos enfocamos en la
siguiente parte:

\begin{lstlisting}
diff --git a/ssl/t1_lib.c b/ssl/t1_lib.c
index a2e2475..bcb99b8 100644
--- a/ssl/t1_lib.c
+++ b/ssl/t1_lib.c
@@ -3969,16 +3969,20 @@ tls1_process_heartbeat(SSL *s)
 	unsigned int payload;
 	unsigned int padding = 16; /* Use minimum padding */

-	/* Read type and payload length first */
-	hbtype = *p++;
-	n2s(p, payload);
-	pl = p;
-
 	if (s->msg_callback)
 		s->msg_callback(0, s->version, TLS1_RT_HEARTBEAT,
 			&s->s3->rrec.data[0], s->s3->rrec.length,
 			s, s->msg_callback_arg);

+	/* Read type and payload length first */
+	if (1 + 2 + 16 > s->s3->rrec.length)
+		return 0; /* silently discard */
+	hbtype = *p++;
+	n2s(p, payload);
+	if (1 + 2 + payload + 16 > s->s3->rrec.length)
+		return 0; /* silently discard per RFC 6520 sec. 4 */
+	pl = p;
+
 	if (hbtype == TLS1_HB_REQUEST)
 		{
 		unsigned char *buffer, *bp;
\end{lstlisting}

Luego es cuestión de crear una firma de la función
\textit{tls1\_process\_heartbeat} en su vulnerable. Si bien esto no es algo
trivial, es un problema cotidiano en la práctica, y no nos embarcamos en éste
durante el proyecto. Pero a modo de ejemplo, generamos la firma para una
version con símbolos de OpenSSL compilada con GCC en su version más reciente de
Ubuntu 14.04. Para esto, luego de compilar la librería y linkearla
estáticamente a nginx (a modo de ejemplo), utilizamos gdb para encontrar el
código de la función:

\begin{lstlisting}
(gdb) disassemble /r tls1_process_heartbeat
41 56                           push   %r14
41 55                           push   %r13
41 54                           push   %r12
55                              push   %rbp
48 89 fd                        mov    %rdi,%rbp
53                              push   %rbx
48 83 ec 10                     sub    $0x10,%rsp
48 8b 97 80 00 00 00            mov    0x80(%rdi),%rdx
4c 8b a2 30 01 00 00            mov    0x130(%rdx),%r12
41 0f b6 5c 24 01               movzbl 0x1(%r12),%ebx
41 0f b6 44 24 02               movzbl 0x2(%r12),%eax
45 0f b6 2c 24                  movzbl (%r12),%r13d
c1 e3 08                        shl    $0x8,%ebx
09 c3                           or     %eax,%ebx
48 8b 87 98 00 00 00            mov    0x98(%rdi),%rax
48 85 c0                        test   %rax,%rax
74 23                           je     0x4b6252 <tls1_process_heartbeat+98>
44 8b 82 24 01 00 00            mov    0x124(%rdx),%r8d
48 8b 97 a0 00 00 00            mov    0xa0(%rdi),%rdx
49 89 f9                        mov    %rdi,%r9
4c 89 e1                        mov    %r12,%rcx
48 89 14 24                     mov    %rdx,(%rsp)
ba 18 00 00 00                  mov    $0x18,%edx
8b 37                           mov    (%rdi),%esi
31 ff                           xor    %edi,%edi
ff d0                           callq  *%rax
66 41 83 fd 01                  cmp    $0x1,%r13w
74 4f                           je     0x4b62a8 <tls1_process_heartbeat+184>
31 c0                           xor    %eax,%eax
66 41 83 fd 02                  cmp    $0x2,%r13w
74 0e                           je     0x4b6270 <tls1_process_heartbeat+128>
48 83 c4 10                     add    $0x10,%rsp
5b                              pop    %rbx
5d                              pop    %rbp
41 5c                           pop    %r12
41 5d                           pop    %r13
41 5e                           pop    %r14
c3                              retq
90                              nop
83 fb 12                        cmp    $0x12,%ebx
41 0f b6 54 24 03               movzbl 0x3(%r12),%edx
41 0f b6 4c 24 04               movzbl 0x4(%r12),%ecx
75 e1                           jne    0x4b6262 <tls1_process_heartbeat+114>
8b b5 a0 02 00 00               mov    0x2a0(%rbp),%esi
c1 e2 08                        shl    $0x8,%edx
09 ca                           or     %ecx,%edx
39 d6                           cmp    %edx,%esi
75 d2                           jne    0x4b6262 <tls1_process_heartbeat+114>
83 c6 01                        add    $0x1,%esi
c7 85 9c 02 00 00 00 00 00 00   movl   $0x0,0x29c(%rbp)
89 b5 a0 02 00 00               mov    %esi,0x2a0(%rbp)
eb bd                           jmp    0x4b6262 <tls1_process_heartbeat+114>
0f 1f 00                        nopl   (%rax)
44 8d 73 13                     lea    0x13(%rbx),%r14d
ba 14 0a 00 00                  mov    $0xa14,%edx
be 48 1a 6c 00                  mov    $0x6c1a48,%esi
44 89 f7                        mov    %r14d,%edi
e8 72 45 03 00                  callq  0x4ea830 <CRYPTO_malloc>
49 89 c5                        mov    %rax,%r13
c6 00 02                        movb   $0x2,(%rax)
89 d8                           mov    %ebx,%eax
49 8d 4d 03                     lea    0x3(%r13),%rcx
c1 e8 08                        shr    $0x8,%eax
49 8d 74 24 03                  lea    0x3(%r12),%rsi
48 89 da                        mov    %rbx,%rdx
41 88 45 01                     mov    %al,0x1(%r13)
41 88 5d 02                     mov    %bl,0x2(%r13)
48 89 cf                        mov    %rcx,%rdi
e8 eb 53 17 00                  callq  0x62b6d0 <memcpy>
48 8d 3c 18                     lea    (%rax,%rbx,1),%rdi
be 10 00 00 00                  mov    $0x10,%esi
e8 ed a6 05 00                  callq  0x5109e0 <RAND_pseudo_bytes>
44 89 f1                        mov    %r14d,%ecx
4c 89 ea                        mov    %r13,%rdx
be 18 00 00 00                  mov    $0x18,%esi
48 89 ef                        mov    %rbp,%rdi
e8 4a 7b 02 00                  callq  0x4dde50 <ssl3_write_bytes>
85 c0                           test   %eax,%eax
78 56                           js     0x4b6360 <tls1_process_heartbeat+368>
48 8b 85 98 00 00 00            mov    0x98(%rbp),%rax
48 85 c0                        test   %rax,%rax
74 2a                           je     0x4b6340 <tls1_process_heartbeat+336>
48 8b 95 a0 00 00 00            mov    0xa0(%rbp),%rdx
45 89 f0                        mov    %r14d,%r8d
49 89 e9                        mov    %rbp,%r9
41 81 e0 ff ff 01 00            and    $0x1ffff,%r8d
4c 89 e9                        mov    %r13,%rcx
bf 01 00 00 00                  mov    $0x1,%edi
48 89 14 24                     mov    %rdx,(%rsp)
ba 18 00 00 00                  mov    $0x18,%edx
8b 75 00                        mov    0x0(%rbp),%esi
ff d0                           callq  *%rax
4c 89 ef                        mov    %r13,%rdi
e8 48 48 03 00                  callq  0x4eab90 <CRYPTO_free>
48 83 c4 10                     add    $0x10,%rsp
31 c0                           xor    %eax,%eax
5b                              pop    %rbx
5d                              pop    %rbp
41 5c                           pop    %r12
41 5d                           pop    %r13
41 5e                           pop    %r14
c3                              retq
66 0f 1f 84 00 00 00 00 00      nopw   0x0(%rax,%rax,1)
4c 89 ef                        mov    %r13,%rdi
89 44 24 0c                     mov    %eax,0xc(%rsp)
e8 24 48 03 00                  callq  0x4eab90 <CRYPTO_free>
8b 44 24 0c                     mov    0xc(%rsp),%eax
e9 ed fe ff ff                  jmpq   0x4b6262 <tls1_process_heartbeat+114>
End of assembler dump.
\end{lstlisting}

Una vez obtenido esto, podemos utilizar memsearch.FindBytesSequence para
encontrar el código maquina arriba listado en hexadecimal dentro de cada
proceso. Es importante aclarar que esta firma dista de ser óptima, y está atada
a una sola version de OpenSSL y una sola configuración de compilación. Usar
memsearch.FindRegexpMatch puede ayudar a mejorar esto, pero la necesidad o no
de hacerlo dependerá de las políticas de configuration management de la
organización que esté utilizando MASCHE.\\

Se provee justo a este informe una maquina virtual que cuenta con versiones de
OpenSSL y nginx vulnerables a Heartbleed para recrear todo lo acá explicado.\\

